\chapter{Professional Considerations}

To ensure the integrity of this project it is important to comply with the British Society of Computing (BSC) Code of Conduct. Each of the four sections of the BSC Code of Conduct will be discussed:

	\section{Public Interest}
	
	From the Code of Conduct: \cite{BCSCodeOfConduct}\\
\textit{	You shall:	
	\begin{itemize}
		\item Have due regard for public health, privacy, security and well being of others and the environment.
		\item Have due regard for the legitimate rights of Third Parties*.
		\item Conduct your professional activities without discrimination on the grounds of sex, sexual orientation, marital status, nationality, colour, race, ethnic origin, religion, age or disability, or of any other condition or requirement  
		\item Promote equal access to the benefits of IT and seek to promote the inclusion of all sectors in society wherever opportunities arise.
	\end{itemize}
	}
	
	This does not apply to this project as no user information will be collected and the project does not involve interaction with any Third Parties

	\section{Professional Competence and Integrity}
	
	From the Code of Conduct: \cite{BCSCodeOfConduct}\\
	
\textit{	You shall:	
	\begin{itemize}
		\item Only undertake to do work or provide a service that is within your professional competence.
		\item NOT claim any level of competence that you do not possess. 
		\item Develop your professional knowledge, skills and competence on a continuing basis, maintaining awareness of technological developments, procedures, and standards that are relevant to your field.
		\item Ensure that you have the knowledge and understanding of Legislation* and that you comply with such Legislation, in carrying out your professional responsibilities.
		\item Respect and value alternative viewpoints and, seek, accept and offer honest criticisms of work.
		\item Avoid injuring others, their property, reputation, or employment by false or malicious or negligent action or inaction.
		\item Reject and will not make any offer of bribery or unethical inducement.		 
	\end{itemize}}
	
	This project will adhere to this section of the Code of Conduct and the author has been assessed to ensure they have the correct level of competency to undertake the project. Once the project has been completed the author will meet with the relevant supervisors to discuss the decisions taken during the project.

	\section{Duty to Relevant Authority}
	
	From the Code of Conduct: \cite{BCSCodeOfConduct}	\\
	
\textit{	You shall:
	\begin{itemize}
		\item Carry out your professional responsibilities with due care and diligence in accordance with the Relevant Authority’s requirements whilst exercising your professional judgement at all times.
		\item Seek to avoid any situation that may give rise to a conflict of interest between you and your Relevant Authority.
		\item Accept professional responsibility for your work and for the work of colleagues who are defined in a given context as working under your supervision.
		\item NOT disclose or authorise to be disclosed, or use for personal gain or to benefit a third party, confidential information except with the permission of your Relevant Authority, or as required by Legislation
		\item NOT misrepresent or withhold information on the performance of products, systems or services (unless lawfully bound by a duty of confidentiality not to disclose such information), or take advantage of the lack of relevant knowledge or inexperience of others.
	\end{itemize}}
	
	The author will adhere to this section of the Code of Conduct. 

	\section{Duty to Profession}
	
	From the Code of Conduct: \cite{BCSCodeOfConduct}\\
	
\textit{	You shall:
	\begin{itemize}
		\item Accept your personal duty to uphold the reputation of the profession and not take any action which could bring the profession into disrepute.
		\item Seek to improve professional standards through participation in their development, use and enforcement.
		\item Uphold the reputation and good standing of BCS, the Chartered Institute for IT.
		\item Act with integrity and respect in your professional relationships with all members of BCS and with members of other professions with whom you work in a professional capacity.
		\item Notify BCS if convicted of a criminal offence or upon becoming bankrupt or disqualified as a Company Director and in each case give details of the relevant jurisdiction.
		\item Encourage and support fellow members in their professional development
	\end{itemize}}
	
	The author will adhere to this section of the Code of Conduct.