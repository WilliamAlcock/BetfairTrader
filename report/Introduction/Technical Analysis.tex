\section{Technical Analysis}
	Technical Analysis assumes that the forces of supply and demand are already incorporated in the market price. A higher volume of unmatched bets in the market to back a horse as opposed to lay will drive the price down as traders place back bets at lower prices in an effort to get matched. The underlying premise is that if the price of a selection is moving lower then the trader should back the selection with a view to laying it at a lower price, making a profit from following the markets trend.\\ 
	
	Quantitative technical analysis is the application of mathematical indicators on previous price data to predict a future trend. They are commonly based on the high, low and closing price and volume traded over intervals of equal duration in the market data. A simple example could be a moving average of closing prices at 1 minute intervals since a market opened. If the current price is above the moving average it could be predicted to rise, if it is below the moving average it could be predicted to fall. Studies have shown that quantitative technical analysis can be effective when used by an automated agent to predict market direction on financial markets\cite{schoreels2004agent}.\\
	
	In Technical Analysis of the Financial Markets: A Comprehensive Guide to Trading Methods and Applications\cite{murphy1999technical} John Murphy states that 'One of the great strengths of technical analysis is its adaptability to virtually any trading medium and time dimension'.\\ 
	
	The second part of this project will use quantitative technical analysis and machine learning to predict the future direction of a market from historical price data.